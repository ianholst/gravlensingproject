\documentclass[10pt]{article}
\usepackage[margin=1in]{geometry}
\usepackage{parskip}
\usepackage{microtype}
\usepackage{natbib}
\usepackage{graphicx}
\usepackage{amsmath}
% \usepackage{txfonts}

\graphicspath{ {figures/} }
\linespread{1.3}


\begin{document}

\title{Distinguishing NFW and Isothermal Density Profiles with Weak Gravitational Lensing}
\author{Ian Holst and Doyee Byun}
\date{\today}
\maketitle

\begin{abstract}
We examine the feasibility of distinguishing NFW and cored isothermal density profiles using weak gravitational lensing shear.

We model lenses in the two different profiles, as well as background galaxies to be lensed.
Analyzing the simulated data of these lensed galaxies gives us insight into how we can distinguish the differences between the two profiles.
This method is expected to be helpful in the analysis of real observation data in the future.
\end{abstract}


\section{Introduction}
Gravitational lensing allows us to investigate the matter density distributions of cosmic objects by observing the characteristic light distortions imparted on other objects in the background. The principle is applicable on a large range of scales, from all-sky mass maps based on cosmic microwave background shear, to studies of individual galaxies and clusters. In particular, by observing the coherent distortions of many background galaxies, the shear profile, and consequently the density profile of a foreground halo lens may be determined.

Two commonly used density profiles in the field are the isothermal profile and the Navarro-Frenk-White (NFW) profile. The NFW profile has been found to fit simulated dark matter halos very well \citep{}, while the isothermal profile results from an idealized isothermal collapse \citep{}. Both profiles extend to infinity

It is of great interest to determine the general forms of halo density profiles
since this tells us about structure formation and the nature of dark matter
\citep{}
Lensing offers one of the few ways to probe the distribution of all matter, not just baryonic matter.
rotation curve present difficulties (https://arxiv.org/pdf/astro-ph/0201352.pdf)

In this paper, we will investigate the distinguishability of NFW and isothermal profiles
given galaxy-galaxy lensing data
The goal of this project is to devise a method to analyze lensed galaxy cluster data and find which density profile is more probable between the isothermal and NFW profiles.
In order to do so, we generate simulated data sets and analyze them to find the more probable profiles they would fit into.
This analysis method is expected to be usable in determining the characteristics of real observed data as well.


Outline:
We start by introducing the general lensing characteristics of spherical density profiles, followed by the specifics for cored isothermal and NFW profiles. We then describe
The results


\subsection{Lensing by a spherical lens profile}
Here we define our conventions for various lensing quantities, which mainly conform to those used by \citet{Dodelson2017}. We use the thin lens approximation and assume spherical lens profiles. Consider such a spherically symmetric halo density profile $\rho(r)$ acting as a gravitational lens. This lens halo is located at an angular diameter distance $D_L$ from the observer and we want to find out how it affects a source object behind it at an angular diameter distance $D_S$. Set the $z$ axis to be along the line of sight, and use local cylindrical coordinates $R$, $\phi$, and $z$. The projected surface density $\Sigma$ at radius $R$ on the projected plane is obtained by integrating the density over the entire $z$ axis:
\begin{equation}
\Sigma(R) = \int_{-\infty}^{\infty}{dz\ \rho(\sqrt{R^2 + z^2})}
\end{equation}
A useful related measure in gravitational lensing is the average projected surface density within a radius $R$:
\begin{equation}
\overline{\Sigma}(R) = \frac{1}{\pi R^2} \int_0^{2\pi}{d\phi \int_0^{R}{dR'~\Sigma(R')R'}}
\end{equation}
The critical surface density is an important quantity that marks the typical boundary between what is considered strong and weak lensing:
\begin{equation}
\Sigma_\mathrm{crit} = \frac{c^2}{4\pi G} \frac{D_S}{D_{SL} D_L}
\end{equation}

$D_S$ and $D_L$ are calculated like normal angular diameter distances in a flat expanding universe: $D = \chi(z)/(1 + z)$, where $\chi(z)$ is the comoving distance to redshift $z$. The redshift $z$ of an object, such as a halo lens or a background galaxy, is typically easy to measure. Note that $D_{SL}$ is nominally the distance from the source to the lens, but since it is an angular diameter distance, it is calculated as $D_{SL} = (\chi_S - \chi_L)/(1 + z_S)$.

All relevant angles in this problem are small, so we can assume that $R = D_L \theta$, where $\theta$ is the angular position on the observer's sky, with the halo center at the origin. This may be represented as the vector $\vec{\theta}$, but in the spherically symmetric case, the magnitude $\theta$ is sufficient. Thus we can easily change variables in all lensing quantities from $R$ to $\theta$.

We define the convergence $\kappa$ at angular position $\theta$ as the ratio of surface density to critical surface density. Strong lensing phenomena such as arcs, rings, multiple images, and magnification are dominant when the convergence is greater than 1, or the projected surface density is greater than the critical surface density.
\begin{equation}
\kappa(\theta) = \frac{\Sigma(\theta)}{\Sigma_\mathrm{crit}}
\end{equation}
While the convergence determines how an object is uniformly scaled or magnified by a gravitational lens, another quantity, shear, describes the stretching of images along an axis. Shear is usually expressed as two components, $\gamma_1$ and $\gamma_2$. The tangential shear $\gamma_t$ is the component of stretching in the $\hat{\phi}$ direction. It can be shown that for a spherical lens profile, the only component of shear $\gamma$ should be the tangential shear, $\gamma_t$. For a spherical lens, tangential shear can be shown to be simply related to the convergence and average convergence \citep{Dodelson2017}:
\begin{equation}
\gamma_t(\theta) = \overline{\kappa}(\theta) - \kappa(\theta)
\end{equation}
The tangential shear is decomposed into two components, $\gamma_1$ and $\gamma_2$. $\gamma_1$ represents stretching along the $\theta_x$ and $\theta_y$ axes, and $\gamma_2$ represents stretching along the $\theta_y = \theta_x$ and $\theta_y = - \theta_x$ lines. They can be obtained through a relation with $\gamma_t$ and the cylindrical azimuthal angle $\phi$.
\begin{equation}
\begin{split}
\gamma_1 = -\gamma_t \cos{2\phi}\\
\gamma_2 = -\gamma_t \sin{2\phi}
\end{split}
\end{equation}
The shear magnitude, which is equal to tangential shear, also follows from trigonometric identities:
\begin{equation}
\gamma = \gamma_t = -\gamma_1 \cos{2\phi} -\gamma_2 \sin{2\phi}
\end{equation}

Changes in object shapes due to lensing can be quantified using the ellipticity $\epsilon$. This is most commonly used in weak lensing, where shape changes are best described by simple stretching. Ellipticity is defined using the semimajor axis $a$ and semiminor axis $b$ of an ellipse \citep{Narayan1996}:
\begin{equation}
\epsilon = \frac{a^2 - b^2}{a^2 + b^2}
\end{equation}
Like shear, there are two components of ellipticity, $\epsilon_1$ and $\epsilon_2$, which are analogous to the shear components $\gamma_1$ and $\gamma_2$. The ellipticity that would be induced in an otherwise perfectly circular object is:
\begin{equation}
\epsilon_i = \frac{2 \gamma_i/(1 - \kappa)}{1 + \gamma^2/(1 - \kappa)^2}
\end{equation}
And there is an analogous ellipticity magnitude or tangential ellipticity:
\begin{equation}
\epsilon =  -\epsilon_1 \cos{2\phi} -\epsilon_2 \sin{2\phi}
\end{equation}

Gravitational lensing causes a deflection in the observed image position of background objects. If the true position of the object is $\vec{\beta}$ and it appears at position $\vec{\theta}$, then the deflection angle is defined:
\begin{equation}
\vec{\alpha}(\vec{\theta}) = \overline{\kappa}(\theta)\vec{\theta}
\end{equation}
From this expression, it can be seen that deflection is most significant in the strong lensing regime, where $\kappa > 1$.


\subsection{Cored Isothermal Sphere Profile}
The cored isothermal sphere (CIS) profile is related to the singular isothermal sphere profile, which is often used to describe halos and other collapsed astrophysical objects because of its simple formulation. Unlike the singular isothermal sphere, the cored isothermal sphere does not have a density singularity at its center due to a finite core radius $r_c$. The CIS density profile is defined:

\begin{equation}
\rho_\mathrm{iso}(r) = \frac{\sigma^2}{2\pi G (r^2 + {r_c}^2)}
\end{equation}

where $\sigma^2$ is the velocity dispersion.
https://arxiv.org/pdf/astro-ph/9810164.pdf
% \begin{equation}\Sigma_\mathrm{iso}(R) = \frac{\sigma^2}{2 G \sqrt{R^2 + {r_c}^2}}
% \end{equation}
%
% \begin{equation}\overline{\Sigma}_\mathrm{iso}(R) = \frac{\sigma^2 \left(\sqrt{R^2 + {r_c}^2} - r_c \right)}{G R^2}
% \end{equation}
%
% In terms of angles:

\begin{equation}
\Sigma_\mathrm{iso}(\theta) = \frac{\sigma^2}{2 G D_L \sqrt{\theta^2 + {\theta_c}^2}}
\end{equation}

\begin{equation}
\overline{\Sigma}_\mathrm{iso}(\theta) = \frac{\sigma^2 \left(\sqrt{\theta^2 + {\theta_c}^2} - \theta_c \right)}{G D_L \theta^2}
\end{equation}

\begin{equation}
\gamma_\mathrm{iso}(\theta) = \frac{\sigma^2 \left(\sqrt{\theta^2 + {\theta_c}^2} - \theta_c \right)}{\Sigma_\mathrm{crit} G D_L \theta^2} - \frac{\sigma^2}{2 \Sigma_\mathrm{crit} G D_L \sqrt{\theta^2 + {\theta_c}^2}}
\end{equation}

We switch dependence from $\sigma^2$ to $M_{200}$ with:

\begin{equation}
\sigma^2 = \frac{M_{200} G}{2 \left( r_{200} - r_c \arctan{\left(\frac{r_{200}}{r_c}\right)} \right)}
\end{equation}

This is derived from the definition of $M_{200}$:

\begin{equation}
M_{200} = 200 \rho_\mathrm{crit} \frac{4}{3} \pi {r_{200}}^3
\end{equation}
\begin{equation}
M_{200} = M_\mathrm{enc}(r_{200}) = \frac{2 \sigma^2}{G} \left( r_{200} - r_c \arctan{\left(\frac{r_{200}}{r_c}\right)} \right)
\end{equation}
\begin{equation}
r_{200} = \left( \frac{3 M_{200}}{800 \pi \rho_\mathrm{crit}} \right)^{1/3}
\end{equation}

The ellipticity equations, while not quite elegant, are trivial to calculate from the shear.

\begin{figure}
    \centering
    \includegraphics[width=0.8\textwidth]{isothermalproperties.pdf}
    \caption{Lensing quantities for a CIS profile with $M_{200} = 10^{15} M_\odot$ and $r_c = 10$ kpc.}
    \label{}
\end{figure}


\subsection{Navarro-Frenk-White (NFW) Profile}

\begin{equation}
\rho_\mathrm{NFW}(r) = \frac{\rho_\mathrm{crit} \delta_c}{(r/r_s)\left(1 + r/r_s\right)^2}
\end{equation}

%What redshift is $\rho_\mathrm{crit}$ evaluated at? - at halo redshift - can we use current time?

% \begin{equation}\Sigma_\mathrm{NFW}(R) = \frac{2 \rho_\mathrm{crit} \delta_c r_s}{(R/r_s)^2 - 1} \left(1 - \frac{2}{\sqrt{(R/r_s)^2 - 1}}  \arctan\left(\sqrt{\frac{R/r_s - 1}{R/r_s + 1}} \right) \right)
% \end{equation}
%
%
% \begin{equation}\overline{\Sigma}_\mathrm{NFW}(R) = \frac{4 \rho_\mathrm{crit} \delta_c r_s}{(R/r_s)^2} \left(
%     \frac{2}{\sqrt{(R/r_s)^2 - 1}} ~\arctan\left(\sqrt{\frac{R/r_s - 1}{R/r_s + 1}} \right) + \ln{\left(\frac{R/r_s}{2}\right)}
% \right)
% \end{equation}
%
% In terms of angles:

\begin{equation}
\Sigma_\mathrm{NFW}(\theta) = \frac{2 \rho_\mathrm{crit} \delta_c D_L \theta_s}{(\theta/\theta_s)^2 - 1} \left(1 - \frac{2}{\sqrt{(\theta/\theta_s)^2 - 1}} \arctan\left(\sqrt{\frac{\theta/\theta_s - 1}{\theta/\theta_s + 1}} \right) \right)
\end{equation}

\begin{equation}
\overline{\Sigma}_\mathrm{NFW}(\theta) = \frac{4 \rho_\mathrm{crit} \delta_c D_L \theta_s}{(\theta/\theta_s)^2} \left(
    \frac{2}{\sqrt{(\theta/\theta_s)^2 - 1}} ~\arctan\left(\sqrt{\frac{\theta/\theta_s - 1}{\theta/\theta_s + 1}} \right) + \ln{\left(\frac{\theta/\theta_s}{2}\right)}
\right)
\end{equation}

Similar convention used by \citet{Bartelmann2001}

\begin{equation}
\gamma_\mathrm{NFW}(\theta) = \frac{\overline{\Sigma}_\mathrm{NFW}(\theta) - \Sigma_\mathrm{NFW}(\theta)}{\Sigma_\mathrm{crit}}
\end{equation}

We can calculate ellipticities from tangential shear.

We switch the dependence to $M_{200}$ and $c$ with:

\begin{equation}
\delta_c = \frac{200}{3} \frac{c^3}{\ln(1 + c) - c/(1 + c)}
\end{equation}
\begin{equation}
r_s = \frac{r_{200}}{c}
\end{equation}
\begin{equation}
r_{200} = \left( \frac{3 M_{200}}{800 \pi \rho_\mathrm{crit}} \right)^{1/3}
\end{equation}
\begin{equation}
c = \frac{r_{200}}{r_s}
\end{equation}

\begin{figure}
    \centering
    \includegraphics[width=0.8\textwidth]{nfwproperties.pdf}
    \caption{Lensing quantities for a NFW profile with $M_{200} = 10^{15} M_\odot$ and $c = 10$.}
    \label{}
\end{figure}




\section{Methods}

\subsection{Modelling of Foreground Lens and Background Galaxies}
Based on the calculations shown above, we have modelled singular lenses corresponding to each density profile.
 Background galaxies have been generated via randomization of angular coordinates.

\begin{enumerate}
    \item Consider single foreground lens halo with many background galaxies.
    \begin{itemize}
        \item Start with one NFW halo, then maybe consider more tests.
    \end{itemize}
    \item Construct background galaxies:
    \begin{itemize}
        \item Number density on sky: 50 galaxies/square arcminute (LSST, https://arxiv.org/pdf/1305.0793.pdf)
        \item Intrinsic ellipticity drawn from Gaussian distribution with $\mu=0, \sigma=0.2$ (consistent with https://arxiv.org/pdf/1509.05058.pdf)
        \item Assume they are all that the same distance $D_S$ since this can be determined by redshift (neglecting some noise)
        \item Apply shear and deflection angles to background galaxies to get simulated data: $N$ sets of $\epsilon_1$, $\epsilon_2$, $\theta_1$, $\theta_2$
    \end{itemize}
    \item Recommended values:
    \begin{itemize}
        \item $z_L = 0.3$ (bullet cluster)
        \item $z_S = 1.0$ (Hubble Deep Field) (also consistent with https://arxiv.org/pdf/1509.05058.pdf)
        we use the planck 2015 results
        \item $M_{halo} = 10^{15} M_\odot$
    \end{itemize}
\end{enumerate}

Estimating ellipticity has well-documented issues [cite] due to noise and PSF


\begin{figure}
    \centering
    \includegraphics[width=0.49\textwidth]{isothermalellipticities.pdf}
    \includegraphics[width=0.49\textwidth]{nfwellipticities.pdf}
    \caption{plots}
    \label{}
\end{figure}


\subsection{Data Analysis and Density Profile Determination}
\begin{enumerate}
\item Bin galaxies in annuli by $\theta$ value (use log bins for theta)
\item Calculate mean and standard deviation of ellipticity
\item Attempt to fit both NFW and isothermal profiles, see if the fit is distinguishable
\end{enumerate}

%What theta range should we look at? (5 arcminutes?)

%How to properly do sigma contours?

\section{Results}

\begin{figure}
    \centering
    \includegraphics[width=0.9\textwidth]{comparison.pdf}
    \caption{plots}
    \label{}
\end{figure}

% \begin{figure}
%     \centering
%     \includegraphics[width=0.49\textwidth]{isothermalfitparams.pdf}
%     \includegraphics[width=0.49\textwidth]{nfwfitparams.pdf}
%     \caption{plots}
%     \label{}
% \end{figure}


\section{Conclusions}

Like (COMPARISONS BETWEEN ISOTHERMAL AND NFW MASS PROFILES
FOR STRONG-LENSING GALAXY CLUSTERS), we find that the strong lensing regime provides the most
But also, weak lensing can in fact provide sufficient distinguishability

has been applied to observations before
https://arxiv.org/pdf/astro-ph/9602053.pdf
but not in the context of comparing density profiles using for strong and weak
but this is highly dependent on good data

http://iopscience.iop.org/article/10.1086/590049/pdf looked at NFW vs isothermal in strong lensing with a focus on arcs, and found differing levels of distinguishibility for different lenses.

https://arxiv.org/pdf/1101.0650.pdf - our results don't match

ignore cosmic shear (https://journals.aps.org/prd/pdf/10.1103/PhysRevD.70.023008)

From our data analysis, we were able to find the more probable density profile of a simulated data set.
Since the simulated data is based on the NFW and isothermal profile models, we were able to decisively distinguish between the two.
When analyzing real observed data, we expect the probability ratios to be relatively more even.
Future goals include the use of these methods to analyze observed data of galaxy clusters to find the density profile of their lenses.

%Where to go from here?


\bibliographystyle{plainnat}
\bibliography{references}

\end{document}
