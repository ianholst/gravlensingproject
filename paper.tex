\documentclass[10pt]{article}
\usepackage[margin=1in]{geometry}
\usepackage{parskip}
\usepackage{microtype}
\usepackage{natbib}
\usepackage{graphicx}
% \usepackage{txfonts}

\graphicspath{ {figures/} }


\begin{document}

\title{Distinguishing NFW and Isothermal Density Profiles with Weak Gravitational Lensing}
\author{Ian Holst and Doyee Byun}
\date{\today}
\maketitle

\begin{abstract}
We examine the feasibility of distinguishing NFW and cored isothermal density profiles using weak gravitational lensing shear.
 To do so, we use Python to model lenses in the two different profiles, as well as background galaxies to be lensed.
 Analyzing the data of these lensed galaxies gives us insight into how we can distinguish the differences between the two profiles.
\end{abstract}

\section{Introduction}
Two commonly used density profiles in weak gravitational lensing are the isothermal profile and the Navarro-Frenk-White (NFW) profile.
We start with introducing the general characteristics of spherical density profiles, followed by cored isothermal and NFW profiles.
We then describe the purpose of our project, to distinguish between NFW and isothermal profiles, in detail.

\subsection{General spherical density profile $\rho(r)$}
We describe the common characteristics of spherical density profiles.
Here we define our conventions for various lensing quantities, which mainly conform to those used by \citet{Dodelson2017}. We use the thin lens approximation and assume spherical lens profiles.

The projected surface density at radius R is defined by
\[\Sigma(R) = \int_{-\infty}^{\infty}{dz\ \rho(\sqrt{R^2 + z^2})}\]

while the average projected surface density within R is
\[\overline{\Sigma}(R) = \frac{1}{\pi R^2} \int_0^{2\pi}{d\phi \int_0^{R}{dR'~\Sigma(R')R'}}\]

The critical surface density is
\[\Sigma_\mathrm{crit} = \frac{c^2}{4\pi G} \frac{D_S}{D_{SL} D_L}\]

Convergence
\[\kappa(\theta) = \frac{\Sigma(\theta)}{\Sigma_\mathrm{crit}}\]

Tangential shear
\[\gamma_t(\theta) = \overline{\kappa}(\theta) - \kappa(\theta)\]
\[\gamma_1 = -\gamma_t \cos{2\phi}\]
\[\gamma_2 = -\gamma_t \sin{2\phi}\]
\[\gamma = \gamma_t = \sqrt{{\gamma_1}^2 + {\gamma_2}^2} = -\gamma_1 \cos{2\phi} -\gamma_2 \sin{2\phi}\]

Deflection angle
\[\vec{\alpha}(\vec{\theta}) = \overline{\kappa}(\theta)\vec{\theta}\]
\[\vec{\beta} = \vec{\theta} - \vec{\alpha} = (1 - \overline{\kappa}(\theta))\vec{\theta} \]

Ellipticity
\[\epsilon_i = \frac{2 \gamma_i/(1 - \kappa)}{1 + \gamma^2/(1 - \kappa)^2}\]
\[\epsilon =  -\epsilon_1 \cos{2\phi} -\epsilon_2 \sin{2\phi}\]

In small angle approximation, any length $R = D_L \theta$.

Prove that spherical density profiles only have tangential shear and ellipticity


\subsection{Cored Isothermal Sphere Profile}

\[\rho_\mathrm{iso}(r) = \frac{\sigma^2}{2\pi G (r^2 + {r_c}^2)}\]

% \[\Sigma_\mathrm{iso}(R) = \frac{\sigma^2}{2 G \sqrt{R^2 + {r_c}^2}}\]
%
% \[\overline{\Sigma}_\mathrm{iso}(R) = \frac{\sigma^2 \left(\sqrt{R^2 + {r_c}^2} - r_c \right)}{G R^2}\]
%
% In terms of angles:

\[\Sigma_\mathrm{iso}(\theta) = \frac{\sigma^2}{2 G D_L \sqrt{\theta^2 + {\theta_c}^2}}\]

\[\overline{\Sigma}_\mathrm{iso}(\theta) = \frac{\sigma^2 \left(\sqrt{\theta^2 + {\theta_c}^2} - \theta_c \right)}{G D_L \theta^2}\]

\[\gamma_\mathrm{iso}(\theta) = \frac{\sigma^2 \left(\sqrt{\theta^2 + {\theta_c}^2} - \theta_c \right)}{\Sigma_\mathrm{crit} G D_L \theta^2} - \frac{\sigma^2}{2 \Sigma_\mathrm{crit} G D_L \sqrt{\theta^2 + {\theta_c}^2}}\]

We switch dependence from $\sigma^2$ to $M_{200}$ with:

\[\sigma^2 = \frac{M_{200} G}{2 \left( r_{200} - r_c \arctan{\left(\frac{r_{200}}{r_c}\right)} \right)}\]

This is derived from the definition of $M_{200}$:

\[M_{200} = 200 \rho_\mathrm{crit} \frac{4}{3} \pi {r_{200}}^3\]
\[M_{200} = M_\mathrm{enc}(r_{200}) = \frac{2 \sigma^2}{G} \left( r_{200} - r_c \arctan{\left(\frac{r_{200}}{r_c}\right)} \right)\]
\[r_{200} = \left( \frac{3 M_{200}}{800 \pi \rho_\mathrm{crit}} \right)^{1/3}\]

Ellipticity equations are very ugly but trivial to calculate from the shear.



\subsection{Navarro-Frenk-White (NFW) Profile}

\[\rho_\mathrm{NFW}(r) = \frac{\rho_\mathrm{crit} \delta_c}{(r/r_s)\left(1 + r/r_s\right)^2}\]

% \[\Sigma_\mathrm{NFW}(R) = \frac{2 \rho_\mathrm{crit} \delta_c r_s}{(R/r_s)^2 - 1} \left(1 - \frac{2}{\sqrt{(R/r_s)^2 - 1}}  \arctan\left(\sqrt{\frac{R/r_s - 1}{R/r_s + 1}} \right) \right)\]
%
%
% \[\overline{\Sigma}_\mathrm{NFW}(R) = \frac{4 \rho_\mathrm{crit} \delta_c r_s}{(R/r_s)^2} \left(
%     \frac{2}{\sqrt{(R/r_s)^2 - 1}} ~\arctan\left(\sqrt{\frac{R/r_s - 1}{R/r_s + 1}} \right) + \ln{\left(\frac{R/r_s}{2}\right)}
% \right)\]
%
% In terms of angles:

\[\Sigma_\mathrm{NFW}(\theta) = \frac{2 \rho_\mathrm{crit} \delta_c D_L \theta_s}{(\theta/\theta_s)^2 - 1} \left(1 - \frac{2}{\sqrt{(\theta/\theta_s)^2 - 1}} \arctan\left(\sqrt{\frac{\theta/\theta_s - 1}{\theta/\theta_s + 1}} \right) \right)\]

\[\overline{\Sigma}_\mathrm{NFW}(\theta) = \frac{4 \rho_\mathrm{crit} \delta_c D_L \theta_s}{(\theta/\theta_s)^2} \left(
    \frac{2}{\sqrt{(\theta/\theta_s)^2 - 1}} ~\arctan\left(\sqrt{\frac{\theta/\theta_s - 1}{\theta/\theta_s + 1}} \right) + \ln{\left(\frac{\theta/\theta_s}{2}\right)}
\right)\]

Similar convention used by \citet{Bartelmann2001}

\[\gamma_\mathrm{NFW}(\theta) = \frac{\overline{\Sigma}_\mathrm{NFW}(\theta) - \Sigma_\mathrm{NFW}(\theta)}{\Sigma_\mathrm{crit}} \]

Can calculate ellipticities from tangential shear.

We switch the dependence to $M_{200}$ and $c$ with:

\[\delta_c = \frac{200}{3} \frac{c^3}{\ln(1 + c) - c/(1 + c)}\]
\[r_s = \frac{r_{200}}{c}\]
\[r_{200} = \left( \frac{3 M_{200}}{800 \pi \rho_\mathrm{crit}} \right)^{1/3}\]
\[c = \frac{r_{200}}{r_s}\]

Estimating ellipticity has well-documented issues [cite] due to noise and PSF

\subsection{Project Purpose}
The goal of this project is to devise a method to analyze lensed galaxy cluster data and find which density profile is more probable between the isothermal and NFW profiles.

\newpage

\section{Methods}
\subsection{Calculation of Shear and Deflection Angles for Each Profile}

\subsection{Modelling of Foreground Lens and Background Galaxies}
\begin{enumerate}
    \item Consider single foreground lens halo with many background galaxies.
    \begin{itemize}
        \item Start with one NFW halo, then maybe consider more tests.
    \end{itemize}
    \item Construct background galaxies:
    \begin{itemize}
        \item Number density on sky: 50 galaxies/square arcminute
        \item Intrinsic ellipticity drawn from Gaussian distribution with $\mu=0, \sigma=0.2$
        \item Assume they are all that the same distance $D_S$ since this can be determined by redshift (neglecting some noise)
    \end{itemize}
    \item Recommended values:
    \begin{itemize}
        \item $z_L = 0.3$
        \item $z_S = 1.0$
        \item $M_{halo} = 10^{15} M_\odot$
    \end{itemize}
\end{enumerate}

\subsection{Generating Simulated Data}
\begin{enumerate}
\item Apply shear and deflection angle to background galaxies, get fake data: $N$ sets of $\epsilon_1$, $\epsilon_2$, $\theta_1$, $\theta_2$
\end{enumerate}

\subsection{Data Analysis and Density Profile Determination}
\begin{enumerate}
\item Bin galaxies in annuli by $\theta$ value (use log bins for theta)
\item Calculate mean and standard deviation of ellipticity
\item Attempt to fit both NFW and isothermal profiles, see if the fit is distinguishable
\end{enumerate}


\subsection{Questions}
\begin{itemize}
    \item How to properly do sigma contours?
    \item What theta range should we look at? (5 arcminutes?)
    \item What redshift is $\rho_\mathrm{crit}$ evaluated at? - at halo redshift - can we use current time?
    \item Using $c = 10$?
    \item Where to go from here?
\end{itemize}


\section{Results and Conclusion}



\bibliographystyle{plainnat}
\bibliography{references}

\end{document}
